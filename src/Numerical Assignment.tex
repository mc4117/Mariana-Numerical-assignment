\documentclass[a4paper,12pt, notitlepage]{article}
\usepackage[margin=2.5cm]{geometry}
\usepackage{listings}
\usepackage[parfill]{parskip}
\setlength{\parskip}{\baselineskip}%
\setlength{\parindent}{0pt}%
\usepackage{amsmath}
\usepackage{amssymb}
\usepackage{graphicx}
\usepackage[justification=centering]{caption}
\usepackage{epstopdf}
\usepackage[usenames, dvipsnames]{color}
\usepackage{chngcntr}
\usepackage{titling}
%\counterwithout{footnote}{chapter}
\usepackage{float}
%\renewcommand\theContinuedFloat{\alph{ContinuedFloat}}
\newcommand\tab[1][0.05cm]{\hspace*{#1}}

\title{Numerical Assignment}
\author{Mariana Clare}
\date{\today}

\begin{document}
	
\maketitle
\thispagestyle{empty}
\section{Defining Shallow Water Equations}
The shallow water equations are
\begin{equation}\label{momentumsw}
\frac{\partial \mathbf{u}}{\partial t} + \mathbf{u}\cdot\nabla\mathbf{u} = - 2\Omega \times\mathbf{u} - g\nabla (h + h_{0})
\end{equation}
\begin{equation}\label{masssw}
\frac{\partial h}{\partial t} + \mathbf{u}\cdot\nabla h = - h \nabla \cdot \mathbf{u}
\end{equation}
where $\mathbf{u}$ is the velocity of the flow, $\Omega$ is the angular velocity of the rotating frame of reference, $g$ is the gravitational acceleration constant, $h$ is the depth of the fluid and $h_{0}$ represents the underlying shape that the fluid is flowing over (as defined in \cite{MPE textbook}).

The first equation (\ref{momentumsw}) represents the conservation of momentum and the second equation (\ref{masssw}) represents the conservation of mass. Both equations have a $u\cdot\nabla$ which represents advection.

In order to solve these equations numerically, they first need to be linearised about the state $u = 0$ and $h = H$ ie.
\begin{eqnarray}\nonumber
\mathbf{u} =  & \mathbf{\hat{u}}
 \\   \nonumber
h = &  H + \hat{h} .
\end{eqnarray}
where $H$ is the average fluid depth. If we further assume that $h_{0}$ is constant, this gives
\begin{equation}
\frac{\partial \mathbf{\hat{u}}}{\partial t} = 2 \Omega \times \mathbf{\hat{u}} - g \nabla \hat{h}
\end{equation}
\begin{equation}
\frac{\partial \hat{h}}{\partial t} = - H \nabla \cdot \mathbf{\hat{u}}
\end{equation}

These equations can be simplified further by assuming there is no rotation and taking the one-dimensional form. Dropping the $\hat{}$ notation this gives
\begin{equation}\label{linearisedsw1}
\frac{\partial u}{\partial t} = - g \frac{\partial h}{\partial x}
\end{equation}
\begin{equation}\label{linearisedsw2}
\frac{\partial h}{\partial t} = - H \frac{\partial u}{\partial x}.
\end{equation}
This report will seek to solve these one-dimensional lineriased shallow water equations numerically. Note that throughout this report we will assume for simplicity that $u$ and $h$ have periodic boundary conditions.

\section{Numerical Methods}

The first numerical method we use to attempt to solve (\ref{linearisedsw1}) and (\ref{linearisedsw2}) is a simple co-located forward-backward in time and centred in space scheme. As in \cite{MPE textbook}, we consider the scheme to be forward in $u$ and backward in $h$. Co-located means we define the velocity and the height at the same location in space on the meshgrid and these schemes are also referred to as A-grid or unstaggered schemes.

This scheme can be written as 
\begin{equation} \label{FTCSAgrid}
\frac{u_{j}^{(n+1)} - u_{j}^{(n)}}{\Delta t} = -g \frac{h_{j+1}^{(n)} - h_{j-1}^{(n)}}{2\Delta x}
\end{equation}
\begin{equation}\label{BTCSAgrid}
\frac{h_{j}^{(n+1)} - h_{j}^{(n)}}{\Delta t} = -H \frac{u_{j+1}^{(n+1)} - u_{j-1}^{(n+1)}}{2\Delta x}
\end{equation}
where $h_{j}^{(n)} = h(x_{j}, t^{(n)})$, $u_{j}^{(n)} = h(u_{j}, t^{(n)})$, $x_{j} = j\Delta x$ and $t^{(n)} = n\Delta t$. 

We can determine the order of accuracy of this scheme, by using Taylor series. We note the following Taylor series expansions:

\begin{equation}\label{ujn+1}
u_{j}^{(n+ 1)} = u_{j}^{(n)} + \Delta t \frac{\partial u_{j}^{(n)}}{\partial t} + \frac{(\Delta t)^{2}}{2}\frac{\partial^{2} u_{j}^{(n)}}{\partial t^{2}} + O((\Delta t)^{3})
\end{equation}
\begin{equation}\label{hj+-1n}
h_{j \pm 1}^{(n)} = h_{j}^{(n)} \pm \Delta x  \frac{\partial h_{j}^{(n)}}{\partial x} + \frac{(\Delta x)^{2}}{2}\frac{\partial^{2} h_{j}^{(n)}}{\partial x^{2}} \pm \frac{(\Delta x)^{3}}{6}\frac{\partial^{3} h_{j}^{(n)}}{\partial x^{3}} + O((\Delta x)^{4})
\end{equation}

Substuting these expansions into the scheme (\ref{FTCSAgrid}), we find that 

\begin{equation}
\frac{\partial u_{j}^{(n)}}{\partial t} + O(\Delta t) =  -g \frac{\partial h_{j}^{(n)}}{\partial x} + O((\Delta x)^{2})
\end{equation} 
and therefore the scheme is first order accurate in time and second order accurate in space. (Note the same order of accuracy is obtained by performing the same analysis with (\ref{BTCSAgrid})).

In order to find when this scheme is stable, we use a von-neumann stability analysis. As in \cite{MPE textbook}, we assume that $u$ and $h$ have wave-like solutions with an amplification factor $A$ and wavenumber $k$:
\begin{equation} \label{wavelikeu}
u  =  \mathbb{U}  A^{n} e^{ikj\Delta x}
\end{equation}
\begin{equation} \label{wavelikeh}
h  =  \mathbb{H} A^{n} e^{ikj\Delta x}
\end{equation}
for constant $\mathbb{U}$ and $\mathbb{H}$.

Further if we define the courant number 
\begin{equation}\label{courantnumber}
c = \frac{\sqrt{gH}\Delta t}{\Delta x}
\end{equation}

then substituting these solutions into (\ref{FTCSAgrid}) and (\ref{BTCSAgrid}) gives
\begin{equation}
A = 1 - \frac{c^{2}}{2} \sin^{2}(k\Delta x) \pm \frac{ic}{2}\sin(k\Delta x)\sqrt{4 - c^{2}\sin^{2}k\Delta x}.
\end{equation} 
 
Hence as found in \cite{MPE textbook}, when $\lvert c \rvert \leq 2$, $\lvert A \rvert^{2} = 1$ and the scheme is stable, but when $\lvert c \rvert > 2$, the scheme is unstable.
 
We can also find the dispersion relation, because the frequency of the numerical method is 
\begin{equation} \label{frequency}
\omega = \pm \frac{1}{\Delta t} \arctan\bigg(\frac{Im(A)}{Re(A)}\bigg)
\end{equation}
Using the result from \cite{MPE textbook}, 
\begin{equation}
\omega_{n}\Delta x = \pm \frac{2}{c} \sin^{-1} \bigg(\frac{c}{2}\sin(k\Delta x)\bigg)
\end{equation}
assuming $\sqrt{gH} = 1$. The positive branch of this dispersion relation is plotted in Figure \ref{dispersionfigure} and compared with the dispersion relation of the analytical solution. (The dispersion relation of the analytical solution $\omega = k\sqrt{gH}$ is found by substituting the wave-like solutions (\ref{wavelikeu}) and (\ref{wavelikeh}) into (\ref{linearisedsw1}) and (\ref{linearisedsw2})).


However, we would like to have a method that was stable for all courant numbers. Therefore another method we can use is an implicit method on a co-located grid. As in \cite{MPE textbook}, we look at the forward in time for both $u$ and $h$ and centred in space co-located scheme:
\begin{equation} \label{FTimplicitAgrid1}
\frac{u_{j}^{(n+1)} - u_{j}^{(n)}}{\Delta t} = -g \frac{h_{j+1}^{(n+1)} - h_{j-1}^{(n+1)}}{2\Delta x}
\end{equation}
\begin{equation}\label{FTimplicitAgrid2}
\frac{h_{j}^{(n+1)} - h_{j}^{(n)}}{\Delta t} = -H \frac{u_{j+1}^{(n+1)} - u_{j-1}^{(n+1)}}{2\Delta x}.
\end{equation}

In order to find the values of $h_{j}^{n}$ and $u_{j}^{n}$ at the next time iteration for all $j$, we consider the matrix:

\[
A = \left (
\begin{array}{ccc}
\begin{array}{ccccc}
1 + \frac{c^{2}}{2} & 0 & -\frac{c^{2}}{4} & 0 & 0\\
0& 1 + \frac{c^{2}}{2} & 0 & -\frac{c^{2}}{4} & 0\\
-\frac{c^{2}}{4} & 0& 1 + \frac{c^{2}}{2} & 0 & -\frac{c^{2}}{4}\\
\vdots & & \vdots & & \vdots\\
- \frac{c^{2}}{4} & 0 & 0 & 0 & 0\\
0 & - \frac{c^{2}}{4} & 0 & 0 & 0\\
\end{array}
\begin{array}{c}
\vdots\\ 
\ddots\\
\vdots
\end{array}
\begin{array}{ccccc}
0 & 0 & 0 & - \frac{c^{2}}{4} & 0\\
0 & 0 & 0 & 0 & - \frac{c^{2}}{4}\\
\vdots & & \vdots & & \vdots\\
-\frac{c^{2}}{4}& 0 & 1 + \frac{c^{2}}{2} & 0 & -\frac{c^{2}}{4} \\
0 & -\frac{c^{2}}{4} & 0 & 1 + \frac{c^{2}}{2} & 0\\
0 & 0 & -\frac{c^{2}}{4}& 0 & 1 + \frac{c^{2}}{2}
\end{array}
\end{array}
\right )
\]

where $c$ is the courant number as before. 

We then rewrite the scheme (\ref{FTimplicitAgrid1}) and (\ref{FTimplicitAgrid2}) as the matrix equation $A \mathbf{x} = \mathbf{b}$ where 
\begin{equation}
x_{j} = h_{j}^{(n+1)} \text { and } b_{j} = h_{j}^{(n)} - \frac{c}{2}\sqrt{\frac{H}{g}}(u_{j+1}^{(n)} - u_{j-1}^{(n)}) \tab[1cm] \forall j \in [0, N-1]
\end{equation}
if solving for $h$ and
\begin{equation}
x_{j} = u_{j}^{(n+1)} \text { and } b_{j} = u_{j}^{(n)} - \frac{c}{2}\sqrt{\frac{g}{H}}(h_{j+1}^{(n)} - h_{j-1}^{(n)}) \tab[1cm] \forall j \in [0, N-1]
\end{equation}
if solving for $u$.

Note that the matrix $A$ has dimension $N \times N$ and we consider the first row to be $j = 0$. 

Furthermore, here and throughout this report we are assuming periodic boundary conditions and hence that $u_{0}^{(n)} = u_{N}^{(n)}$ and $h_{0}^{(n)} = h_{N}^{(n)}$ where $N\Delta x$ is the right hand boundary of the $x$-domain. Hence the matrix $A$ has dimension $N \times N$ and we consider the first row to be $j = 0$. 


We can determine the order of accuracy of this scheme, as before by using the Taylor series expansions (\ref{ujn+1}) and
\begin{equation}
h_{j \pm 1}^{(n+1)} = h_{j}^{(n)} \pm \Delta x \frac{\partial h_{j}^{(n)}}{\partial x} + \frac{(\Delta x)^{2}}{2} \frac{\partial^{2}h_{j}^{(n)}}{\partial x^{2}} \pm \Delta x \Delta t \frac{\partial^{2}h_{j}^{(n)}}{\partial x\partial t} + \frac{(\Delta t)^{2}}{2}\frac{\partial^{2}h_{j}^{(n)}}{\partial t^{2}} + O((\Delta x)^{3}, {(\Delta t)^{3}}).
\end{equation}
Substituting this into equation (\ref{FTimplicitAgrid1}) we find that
\begin{equation}
\frac{\partial u_{j}^{(n)}}{\partial t} + O(\Delta t) = - g \frac{\partial h_{j}^{(n)}}{\partial x} + O((\Delta x)^{2}).
\end{equation}
Therefore the scheme is first order accurate in time and second order accurate in space as with the explicit method on the co-located grid. (As before a similar result can be obtained by substituting in to (\ref{FTimplicitAgrid2})).

In order to find where this method is stable, we use von-neumann stability analysis and assume $u$ and $h$ have wave-like solutions (\ref{wavelikeu}) and (\ref{wavelikeh}). Substituting these solutions into (\ref{FTimplicitAgrid1}) and (\ref{FTimplicitAgrid2}) we find
\begin{equation}
A = \frac{1 \pm i c\sin(k\Delta x)}{1 + c^{2}\sin^{2}(k\Delta x)}
\end{equation}
Therefore 
\begin{equation}
\lvert A \rvert ^{2} = \frac{1}{1 + c^{2}\sin^{2}(k\Delta x)} < 1  \tab[1cm] \forall k, \Delta x
\end{equation}
and the system is stable everywhere. 

Using (\ref{frequency}) we can find the dispersion relation
\begin{equation}
\omega_{n} \Delta x = \pm\frac{\Delta x}{\Delta t} \arctan(c\sin(k\Delta x)) = \frac{1}{c}  \arctan(c\sin(k\Delta x))
\end{equation}
assuming $\sqrt{gH} = 1$. The positive branch of this dispersion relation is again plotted in Figure \ref{dispersionfigure}.

We can see from this Figure \ref{dispersionfigure} that the analytic solution and numerical solution do not propagate at the same rate. The numeric solution disperses too slowly and in fact at $k = \pi$, the wave is stationary. 

Therefore we seek a scheme which propagates at a more similar speed to the analytic solution. Following \cite{MPE textbook}, we use a staggered grid (sometimes known as a C-grid) instead of a co-located grid. For a staggered grid, we shift $u$ so that it is defined at $x_{j} + \frac{\Delta x}{2}$ and $h$ remains defined at $x_{j}$ (where $x_{j} = j \Delta x$) ie. where $u$ and $h$ are defined alternates in space.

As in \cite{MPE textbook}, we take again the forward-backward in time and centred in space scheme where the scheme is forward in $u$ and backward in $h$, but this time on a staggered grid. This gives us the following scheme:

\begin{equation}\label{FTCSCgrid}
\frac{u_{j+ \frac{1}{2}}^{(n+1)} - u_{j + \frac{1}{2}}^{(n)}}{\Delta t} = -g \frac{h_{j+1}^{(n)} - h_{j}^{(n)}}{\Delta x}
\end{equation}

\begin{equation}\label{BTCSCgrid}
\frac{h_{j}^{(n+1)} - h_{j}^{(n)}}{\Delta t} = -H \frac{u_{j+\frac{1}{2}}^{(n+1)} - u_{j-\frac{1}{2}}^{(n+1)}}{\Delta x}
\end{equation}


We can determine the order of accuracy of this scheme, as before, by using the Taylor series expansions (\ref{hj+-1n}) and 
\begin{equation} \label{uj+1/2n}
u_{j \pm \frac{1}{2}}^{(n)} = u_{j}^{(n)} \pm \frac{\Delta x}{2}\frac{\partial u_{j}^{(n)}}{\partial x} + \frac{(\Delta x)^{2}}{8}\frac{\partial^{2}u_{j}^{n}}{\partial x^{2}} + O({(\Delta x)^{3}}.
\end{equation}
\begin{equation} \label{uj+1/2n+1}
u_{j \pm \frac{1}{2}}^{(n + 1)} = u_{j}^{(n)} \pm \frac{\Delta x}{2}\frac{\partial u_{j}^{(n)}}{\partial x} + \Delta t \frac{\partial u_{j}^{(n)}}{\partial t} + \frac{(\Delta x)^{2}}{8}\frac{\partial^{2}u_{j}^{n}}{\partial x^{2}} \pm \frac{\Delta t \Delta x}{2}\frac{\partial^{2} u_{j}^{(n)}}{\partial x \partial t} + \frac{(\Delta t)^{2}}{2} \frac{\partial ^{2} u_{j}^{(n)}}{\partial t ^{2}} + O((\Delta x)^{3}, (\Delta t)^{3})
\end{equation}

Substituting these into (\ref{FTCSCgrid}), we find that 
\begin{equation}
\frac{\partial u_{j}^{(n)}}{\partial t} + O(\Delta t) =  -g \frac{\partial h_{j}^{(n)}}{\partial x} + O((\Delta x)^{2})
\end{equation} 
and therefore the scheme is first order accurate in time and second order accurate in space. (Note as before the same order of accuracy is obtained by performing the same analysis with (\ref{BTCSCgrid})).

In order to find where this method is stable, we use von-neumann stability analysis and assume $u$ and $h$ have wave-like solutions (\ref{wavelikeu}) and (\ref{wavelikeh}). Substituting these solutions into (\ref{FTCSCgrid}) and (\ref{BTCSCgrid}) we find
\begin{equation}
A = 1 - 2c^{2}\sin^{2}\bigg(\frac{k\Delta x}{2}\bigg) \pm 2ic\sin\bigg(\frac{k\Delta x}{2}\bigg) \sqrt{1 - c^{2}\sin^{2}(\frac{k\Delta x}{2}\bigg)}
\end{equation}
Therefore if $\lvert c \rvert \leq 1$, this scheme is stable, but if $\lvert c \rvert > 1$ the scheme is unstable.

Using (\ref{frequency}) we can find the dispersion relation
\begin{equation}
	\omega_{n} \Delta x = \pm\frac{2\Delta x}{\Delta t} \arcsin\bigg(c\sin\bigg(\frac{k\Delta x}{2}\bigg)\bigg) = \frac{2}{c} \arcsin\bigg(c\sin\bigg(\frac{k\Delta x}{2}\bigg)\bigg) 
\end{equation}
assuming $\sqrt{gH} = 1$. The positive branch of this dispersion relation is again plotted in Figure \ref{dispersionfigure}. 

Although this solution is still dispersive we can see from the Figure that $\omega$ of this numeric scheme is much closer to $\omega$ of the analytic solution.

However we again have the problem that the solution is unstable. Therefore the final numerical scheme we will look at is the semi-implicit scheme on a staggered grid outlined in \cite{semi-implicit}.

The scheme used in \cite{semi-implicit} is the theta-method:

\begin{equation}
\frac{u_{j + \frac{1}{2}}^{(n + 1)} - u_{j + \frac{1}{2}}^{(n)}}{\Delta t} = -\frac{g}{\Delta x} \bigg(\theta (h_{j + 1}^{(n+ 1)} - h_{j}^{(n+ 1)}) + (1 - \theta) (h_{j + 1}^{(n)} - h_{j}^{(n)})\bigg)
\end{equation}
\begin{equation}
\frac{h_{j}^{(n + 1)} - h_{j}^{(n)}}{\Delta t} = -\frac{H}{\Delta x} \bigg(\theta (u_{j + \frac{1}{2}}^{(n+ 1)} - u_{j - \frac{1}{2}}^{(n+ 1)}) + (1 - \theta) (u_{j + \frac{1}{2}}^{(n)} - u_{j - \frac{1}{2}}^{(n)})\bigg)
\end{equation}

For simplicity we have taken $\theta = \frac{1}{2}$ and are therefore using the Crank-Nicholson method centred in space on a staggered grid:

\begin{equation}\label{semiimplicit1}
\frac{u_{j + \frac{1}{2}}^{(n + 1)} - u_{j + \frac{1}{2}}^{(n)}}{\Delta t} = -\frac{g}{2\Delta x} \bigg((h_{j + 1}^{(n+ 1)} - h_{j}^{(n+ 1)}) + (h_{j + 1}^{(n)} - h_{j}^{(n)})\bigg)
\end{equation}
\begin{equation}\label{semiimplicit2}
\frac{h_{j}^{(n + 1)} - h_{j}^{(n)}}{\Delta t} = -\frac{H}{2\Delta x} \bigg((u_{j + \frac{1}{2}}^{(n+ 1)} - u_{j - \frac{1}{2}}^{(n+ 1)}) + (u_{j + \frac{1}{2}}^{(n)} - u_{j - \frac{1}{2}}^{(n)})\bigg)
\end{equation}

In order to find the values of $h_{j}^{n}$ and $u_{j}^{n}$ at the next time iteration for all $j$, we consider the matrix:

\[
A = \left (
\begin{array}{ccc}
\begin{array}{ccc}
1 + \frac{c^{2}}{2} & -\frac{c^{2}}{4} & 0\\
-\frac{c^{2}}{4}& 1 + \frac{c^{2}}{2} & -\frac{c^{2}}{4} \\
\vdots & \vdots & \vdots\\
0 & 0  & 0 \\
- \frac{c^{2}}{4}  & 0  & 0 
\end{array}
\begin{array}{c}
\vdots\\ 
\ddots\\
\vdots
\end{array}
\begin{array}{ccc}
0  & 0  &  - \frac{c^{2}}{4}\\
0  & 0& 0\\
\vdots & \vdots & \vdots\\
-\frac{c^{2}}{4}& 1 + \frac{c^{2}}{2} & -\frac{c^{2}}{4} \\
0 & -\frac{c^{2}}{4} & 1 + \frac{c^{2}}{2}
\end{array}
\end{array}
\right )
\]
where $c$ is the courant number as before. 

We then rewrite the scheme (\ref{semiimplicit1}) and (\ref{semiimplicit2}) as the matrix equation $A \mathbf{x} = \mathbf{b}$.
\begin{equation}
x_{j} = h_{j}^{(n+1)} \text { and } b_{j} = -c\sqrt\frac{g}{H}(h_{j + 1}^{(n)} - h_{j}^{n}) + \frac{c^{2}}{4} u_{j + \frac{3}{2}}^{(n)} + (1 - \frac{c^{2}}{2})u_{j + \frac{1}{2}}^{(n)} + \frac{c^{2}}{4} u_{j - \frac{1}{2}}^{(n)} \tab[1cm] \forall j \in [0, N-1]
\end{equation}
if solving for $h$ and
\begin{equation}
x_{j} = u_{j+ \frac{1}{2}}^{(n+1)} \text { and } b_{j} = -c\sqrt\frac{H}{g}(u_{j + \frac{1}{2}}^{(n)} - u_{j - \frac{1}{2}}^{n}) + \frac{c^{2}}{4} h_{j + 1}^{(n)} + (1 - \frac{c^{2}}{2})h_{j}^{(n)} + \frac{c^{2}}{4} h_{j - 1}^{(n)} \tab[1cm] \forall j \in [0, N-1]
\end{equation}
if solving for $u$.

Note that the matrix $A$ has dimension $N \times N$ and we consider the first row to be $j = 0$.

Furthermore as before we have assumed periodic boundary conditions and hence that $u_{\frac{1}{2}}^{(n)} = u_{N + \frac{1}{2}}^{(n)}$ and $h_{0}^{(n)} = h_{N}^{(n)}$ where $N\Delta x$ is the right hand boundary of the $x$-domain. 

We can determine the order of accuracy of this scheme, as before by using the Taylor series expansions (\ref{uj+1/2n}) and (\ref{uj+1/2n+1}) and
\begin{equation}
h_{j}^{(n+ 1)} = h_{j}^{(n)} + \Delta t \frac{\partial h_{j}^{(n)}}{\partial t} + \frac{(\Delta t)^{2}}{2}\frac{\partial^{2} h_{j}^{(n)}}{\partial t^{2}} + O((\Delta t)^{3}).
\end{equation}

Substituting these into (\ref{semiimplicit2}) we find that
\begin{equation}
\frac{\partial h_{j}^{(n)}}{\partial t} + O((\Delta t)^{2}) = - H \frac{\partial u_{j}^{(n)}}{\partial x} + O((\Delta x)^{2}) 
\end{equation}
and therefore the scheme is second order accurate in time and second order accurate in space which is better than any of the other schemes we have looked at so far. (Note as before the same order of accuracy is obtained by performing the same analysis with (\ref{semiimplicit1})).

As before in order to find where this method is stable, we use von-neumann stability analysis and assume $u$ and $h$ have wave-like solutions (\ref{wavelikeu}) and (\ref{wavelikeh}). Substituting these solutions into (\ref{semiimplicit1}) and (\ref{semiimplicit2}) we find:

\begin{equation}
A = \frac{2 - 2c^{2}\sin^{2}(\frac{k\Delta x}{2}) \pm 4ic\sin(\frac{k\Delta x}{2})}{2 + 2 c^{2}\sin^{2}(\frac{k\Delta x}{2})}
\end{equation}

$\lvert A \rvert^{2} = 1$ and therefore the scheme is stable and undamping $\forall k$.

Using (\ref{frequency}) we can find the dispersion relation
\begin{equation}
\omega_{n} \Delta x = \pm\frac{\Delta x}{\Delta t} \arctan\bigg(\frac{2 c \sin(\frac{k\Delta x}{2})}{1 - c^{2} \sin^{2}(\frac{k\Delta x}{2})}\bigg) = \pm\frac{1}{c} \arctan\bigg(\frac{2 c \sin(\frac{k\Delta x}{2})}{1 - c^{2} \sin^{2}(\frac{k\Delta x}{2})}\bigg)
\end{equation}
assuming $\sqrt{gH} = 1$. The positive branch of this dispersion relation is again plotted in Figure \ref{dispersionfigure}. 

As with the previous staggered grid scheme, this solution is still dispersive. The dispersion relation for the semi-implicit staggered scheme diverges more from the analytic solution than the explicit staggered scheme but it is still significantly better than either of the co-located schemes. 

\begin{figure}
	\centering
	\includegraphics[width=0.7\textwidth]{dispersion_relations.png}
	\caption{Postive branch of dispersion relation $\omega$ for analytic solution and numerical schemes. Note that as in \cite{MPE textbook}, we have used $c=0.4$ and $\sqrt{gH} = 1$.} \label{dispersionfigure}
\end{figure}

\section{Methodology to test Numerical Methods}
In order to test the properties of the Numerical methods outlined in the above section we have devised a series of tests.

\renewcommand\theContinuedFloat{\alph{ContinuedFloat}}
\begin{figure}
	\begin{minipage}{.5\textwidth}
		\ContinuedFloat*
		%\centering
		\captionsetup{width=0.9\textwidth}
		\captionsetup{justification=centering}
		\includegraphics[width=\textwidth]{initial_condition_cosbell.png}
		\caption{\label{initialconditioncosbell}Initial condition such that $u$ is zero everywhere and $h$ has a bump in the centre with zero either side (hereafter referred to as cosbell)} 
	\end{minipage}
	\begin{minipage}{.5\textwidth}
		\ContinuedFloat
		%\centering
		\captionsetup{width=0.9\textwidth}
		\captionsetup{justification=centering}
		\includegraphics[width=\textwidth]{initial_condition_spike.png}
		\caption{\label{initialconditionspike}Initial condition such that $u$ is zero everywhere and $h$ is zero everywhere apart from one point at the centre where it is one} 
	\end{minipage}
	\begin{center}
		\begin{minipage}{.5\textwidth}
			\ContinuedFloat
			%\centering
			\captionsetup{width=0.9\textwidth}
			\captionsetup{justification=centering}
			\includegraphics[width=\textwidth]{initial_condition_cos.png}
			\caption{\label{initialconditioncos}Initial condition such that $u$ is zero everywhere and $h$ is $\cos(x)$}
		\end{minipage}
	\end{center}
\end{figure}

\subsection{Test 1}
To begin with, we attempt to solve the shallow water equations using a simple initial condition (shown in Figure \ref{initialconditioncosbell}) with the co-located forward-backward explicit scheme.

\subsection{Test 2}
We found using Von-Neumann stability analysis the co-located forward-backward explicit scheme is unstable for courant numbers higher than 2. Therefore we test to see if this is in fact the case.

\subsection{Test 3}
We found using Von-Neumann stability analysis the co-located forward-backward implicit scheme is stable for all courant numbers. Therefore we test to see if this is in fact the case.

\subsection{Test 4}
When looking at the dispersion relation for the co-located grid it seems that the results produced by the scheme may be unphysical. We test this hypothesis by taking a different initial condition (shown in Figure \ref{initialconditionspike}) and plotting the solutions of $u$ and $h$ at multiple time steps for the co-located forward-backward explicit scheme and the co-located forward-backward implicit scheme.

\subsection{Test 5}
Our analysis of the dispersion relations suggests that on a staggered grid the solutions for $u$ and $h$ should be more physical. Therefore we repeat test 3 with the same initial conditions (shown in Figure \ref{initialconditionspike}), but instead using the staggered explicit scheme.

\subsection{Test 6}
So far all the comparisons have been about the properties of the schemes themselves. If the initial condition shown in Figure \ref{initialconditioncos} is chosen it is possible to find an analytic solution of the shallow water equations
\begin{equation}
u = \sin(x)\sin(t)
\end{equation}
\begin{equation}
h = \cos(x)\cos(t).
\end{equation}
If we choose the interval $[-\pi, \pi]$ then these solutions are periodic.

This test will look at the error between the solutions produced by the numerical scheme and the analytic solution. It will also compare these errors to $\Delta x$ and $\Delta t$ to see if the error relationships found by the Taylor series expansions are correct.

\subsection{Test 7}
Finally we compare the computational cost of the four schemes, by comparing the time it takes for each scheme to run, starting from the same initial condition (shown in Figure \ref{initialconditioncos}) and with the same number of time steps and space steps.

\section{Results}
For all the following results, unless otherwise explicitily stated, assume the domain used was $0 \leq x \leq 1$ and that the following parameters were used:

\begin{eqnarray}
c & = & 0.1\\
H & = & 1\\
g & = & 1\\
nx & = & 60\\
nt & = & 100
\end{eqnarray}
where $c$ is the courant number, $H$ is the average fluid depth, $g$ is the gravitational acceleration constant, $nx$ is the number of space-steps in the meshgrid and $nt$ is the number of time-steps.
\subsection{Test 1}
\begin{figure} [H]
	\begin{minipage}{.5\textwidth}
		\ContinuedFloat*
		%\centering
		\captionsetup{width=0.9\textwidth}
		\captionsetup{justification=centering}
		\includegraphics[width=\textwidth]{velocity_colocated_explicit_cosbell.png}
		\caption{\label{velocity_colocated_explicit_cosbell} Velocity at different timesteps using the co-located explicit method using the initial condition $u$ equals $0$ and $h$ is a cosbell (as shown in Figure \ref{initialconditioncosbell}). Here $c$ = } 
	\end{minipage}
	\begin{minipage}{.5\textwidth}
		\ContinuedFloat
		%\centering
		\captionsetup{width=0.9\textwidth}
		\captionsetup{justification=centering}
		\includegraphics[width=\textwidth]{height_colocated_explicit_cosbell.png}
		\caption{\label{height_colocated_explicit_cosbell}    Height at different timesteps using the co-located explicit method using the initial condition $u$ equals $0$ and $h$ is a cosbell (as shown in Figure \ref{initialconditioncosbell}).} 
	\end{minipage}
\end{figure}

\subsection{Test 2}

\begin{figure} [H]
	\begin{minipage}{.5\textwidth}
		\ContinuedFloat*
		%\centering
		\captionsetup{width=0.9\textwidth}
		\captionsetup{justification=centering}
		\includegraphics[width=\textwidth]{velocity_varying_courant_explicit.png}
		\caption{\label{velocity_varying_courant_explicit} Velocity for varying courant numbers for the co-located explicit method} 
	\end{minipage}
	\begin{minipage}{.5\textwidth}
		\ContinuedFloat
		%\centering
		\captionsetup{width=0.9\textwidth}
		\captionsetup{justification=centering}
		\includegraphics[width=\textwidth]{height_varying_courant_explicit.png}
		\caption{\label{height_varying_courant_explicit} Height for varying courant numbers for the co-located explicit method} 
	\end{minipage}
\end{figure}

\subsection{Test 3}

\begin{figure} [H]
	\begin{minipage}{.5\textwidth}
		\ContinuedFloat*
		%\centering
		\captionsetup{width=0.9\textwidth}
		\captionsetup{justification=centering}
		\includegraphics[width=\textwidth]{velocity_varying_courant_implicit.png}
		\caption{\label{velocity_varying_courant_implicit} Velocity for varying courant numbers for the co-located implicit method} 
	\end{minipage}
	\begin{minipage}{.5\textwidth}
		\ContinuedFloat
		%\centering
		\captionsetup{width=0.9\textwidth}
		\captionsetup{justification=centering}
		\includegraphics[width=\textwidth]{height_varying_courant_implicit.png}
		\caption{\label{height_varying_courant_implicit} Height for varying courant numbers for the co-located implicit method} 
	\end{minipage}
\end{figure}

\subsection{Test 4}

\begin{figure} [H]
	\begin{minipage}{.5\textwidth}
		\ContinuedFloat*
		%\centering
		\captionsetup{width=0.9\textwidth}
		\captionsetup{justification=centering}
		\includegraphics[width=\textwidth]{velocity_colocated_explicit_spike.png}
		\caption{\label{velocity_colocated_explicit_spike} CAPTION} 
	\end{minipage}
	\begin{minipage}{.5\textwidth}
		\ContinuedFloat
		%\centering
		\captionsetup{width=0.9\textwidth}
		\captionsetup{justification=centering}
		\includegraphics[width=\textwidth]{height_colocated_explicit_spike.png}
		\caption{\label{height_colocated_explicit_spike} CAPTION} 
	\end{minipage}
\end{figure}

\begin{figure} [H]
	\begin{minipage}{.5\textwidth}
		\ContinuedFloat*
		%\centering
		\captionsetup{width=0.9\textwidth}
		\captionsetup{justification=centering}
		\includegraphics[width=\textwidth]{velocity_colocated_implicit_spike.png}
		\caption{\label{velocity_colocated_implicit_spike} CAPTION} 
	\end{minipage}
	\begin{minipage}{.5\textwidth}
		\ContinuedFloat
		%\centering
		\captionsetup{width=0.9\textwidth}
		\captionsetup{justification=centering}
		\includegraphics[width=\textwidth]{height_colocated_implicit_spike.png}
		\caption{\label{height_colocated_implicit_spike} CAPTION} 
	\end{minipage}
\end{figure}

\subsection{Test 5}

\begin{figure} [H]
	\begin{minipage}{.5\textwidth}
		\ContinuedFloat*
		%\centering
		\captionsetup{width=0.9\textwidth}
		\captionsetup{justification=centering}
		\includegraphics[width=\textwidth]{velocity_staggered_explicit_spike.png}
		\caption{\label{velocity_staggered_explicit_spike} CAPTION} 
	\end{minipage}
	\begin{minipage}{.5\textwidth}
		\ContinuedFloat
		%\centering
		\captionsetup{width=0.9\textwidth}
		\captionsetup{justification=centering}
		\includegraphics[width=\textwidth]{height_staggered_explicit_spike.png}
		\caption{\label{height_staggered_explicit_spike} CAPTION} 
	\end{minipage}
\end{figure}

\subsection{Test 6}
\begin{figure} [H]
	\begin{minipage}{.5\textwidth}
		\ContinuedFloat*
		%\centering
		\captionsetup{width=0.9\textwidth}
		\captionsetup{justification=centering}
		\includegraphics[width=\textwidth]{comparison_with_exact_u.png}
		\caption{\label{exact velocity} CAPTION} 
	\end{minipage}
	\begin{minipage}{.5\textwidth}
		\ContinuedFloat
		%\centering
		\captionsetup{width=0.9\textwidth}
		\captionsetup{justification=centering}
		\includegraphics[width=\textwidth]{comparison_with_exact_h.png}
		\caption{\label{exact height} CAPTION} 
	\end{minipage}
\end{figure}

\begin{figure} [H]
	\begin{minipage}{.5\textwidth}
		\ContinuedFloat*
		%\centering
		\captionsetup{width=0.9\textwidth}
		\captionsetup{justification=centering}
		\includegraphics[width=\textwidth]{error_in_u.png}
		\caption{\label{error_velocity} CAPTION} 
	\end{minipage}
	\begin{minipage}{.5\textwidth}
		\ContinuedFloat
		%\centering
		\captionsetup{width=0.9\textwidth}
		\captionsetup{justification=centering}
		\includegraphics[width=\textwidth]{error_in_h.png}
		\caption{\label{error_height} CAPTION} 
	\end{minipage}
\end{figure}

\subsection{Test 7}

\begin{table}[H]
	\centering
	\begin{tabular}{|c | c|} 
		\hline
		\textbf{Numerical Scheme} & \textbf{Time (3sf)}  \\
		\hline
		Co-located Explicit & 2.48 s \\ 
		\hline
	    Staggered Explicit & 2.52 s \\
		\hline
		Co-located  Implicit & 5.06 s \\
		\hline
		Staggered Semi-implicit & 57.9 s \\
		\hline
	\end{tabular}
\caption{Time taken for each numerical scheme to run for the initial condition that $h$ is $\cos(x)$ and $u$ is $0$ (as shown in Figure \ref{initialconditioncos}) and the number of space steps is 1000 and number of time steps is 1000.}
\label{timingtable}
\end{table}

\section{Conclusions}

\begin{thebibliography}{9}
	\addcontentsline{toc}{part}{Bibliography}
	\bibitem{MPE textbook}
	Cotter, C. and Weller, H., (2018), \textquoteleft Numerical Methods\textquoteright, Ch.5 in Crisan, D (eds.), \textit{Mathematics of Planet Earth: a primer}, World Scientific Publishing Europe Ltd., London.
	\bibitem{semi-implicit}
	Casulli, V. and Cattani, E., (1994), \textquoteleft Stability, Accuracy and Efficiency of a Semi-Implicit Method for Three-Dimensional Shallow Water Flow\textquoteright, \textit{Computers Math. Applic}, \textbf{27(4)}, 99-112.
\end{thebibliography}
\end{document}